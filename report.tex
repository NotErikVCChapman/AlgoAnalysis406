\documentclass{article}

\usepackage{tikz} 
\usetikzlibrary{automata, positioning, arrows} 

\usepackage{amsthm}
\usepackage{amsfonts}
\usepackage{amsmath}
\usepackage{amssymb}
\usepackage{fullpage}
\usepackage{color}
\usepackage{parskip}
\usepackage{hyperref}
  \hypersetup{
    colorlinks = true,
    urlcolor = blue,
    linkcolor= blue,
    citecolor= blue,
    filecolor= blue,
  }
    
\usepackage{listings}
\usepackage[utf8]{inputenc}                                                     
\usepackage[T1]{fontenc}                                                        

\definecolor{dkgreen}{rgb}{0,0.6,0}
\definecolor{gray}{rgb}{0.5,0.5,0.5}
\definecolor{mauve}{rgb}{0.58,0,0.82}

\lstset{frame=tb,
  language=haskell,
  aboveskip=3mm,
  belowskip=3mm,
  showstringspaces=false,
  columns=flexible,
  basicstyle={\small\ttfamily},
  numbers=none,
  numberstyle=\tiny\color{gray},
  keywordstyle=\color{blue},
  commentstyle=\color{dkgreen},
  stringstyle=\color{mauve},
  breaklines=true,
  breakatwhitespace=true,
  tabsize=3
}

\newtheoremstyle{theorem}
  {\topsep}   
  {\topsep}   
  {\itshape}  
  {0pt}       
  {\bfseries} 
  {.}         
  {5pt plus 1pt minus 1pt} 
  {}
\theoremstyle{theorem} 
   \newtheorem{theorem}{Theorem}[section]
   \newtheorem{corollary}[theorem]{Corollary}
   \newtheorem{lemma}[theorem]{Lemma}
   \newtheorem{proposition}[theorem]{Proposition}
\theoremstyle{definition}
   \newtheorem{definition}[theorem]{Definition}
   \newtheorem{example}[theorem]{Example}
\theoremstyle{remark}    
  \newtheorem{remark}[theorem]{Remark}

\title{CPSC-406 Report}
\author{Erik Van Cruyningen  \\ Chapman University}

\date{\today} 

\begin{document}

\maketitle

\begin{abstract}
\end{abstract}

\setcounter{tocdepth}{3}
\tableofcontents

\section{Introduction}\label{intro}

\section{Week by Week}\label{homework}

\subsection{Week 1}

\subsubsection{Homework 1: Introduction to Automata}
\begin{description}
    \item[Problem 1:] \textbf{Characterize All Words to Sum 25 Cents}\\
    All words that end in the accepting state follow a pattern. They can be one of three things: Five fives, three fives and one ten, or one five and two tens. This can be written as:
    $$(5^5) | (5^310) | (5^110^2)$$
    This is every possible combination of valid answers, but answers over 25 cents are rejected.
    
    \item[Problem 2:] \textbf{Create a Regular Expression to express the pattern in the image}\\
    The pattern consists of any number of pushes and pays followed by at least one pay. Following the guidelines in the homework, the regex is:
    \[ (push + pay)^* pay \]
\end{description}

\subsubsection{Homework 1: DFAs and NFAs}
\begin{description}
    \item[Problem 1:] \textbf{Determine whether the following words belong to L1, L2, or L3.}\\
    \begin{align*}
    L1 &:= \{x01y \mid x, y \in \Sigma^*\}\\
    L2 &:= \{w \mid |w| = 2^n, n \in \mathbb{N}\}\\
    L3 &:= \{w \mid |w_0| = |w_1|\}
    \end{align*}
    
    L1 accepts any word that contains ``01'' in it. L2 accepts words where the length is a power of 2. L3 accepts any word where the word is of equivalent length to the word w0.

    \begin{center}
    \begin{tabular}{||c c c c||} 
     \hline
       & L1 & L2 & L3 \\ [0.5ex] 
     \hline\hline
     w1 = 10011 & 1 & 0 & 1 \\
     \hline
     w2 = 100 & 0 & 0 & 0 \\
     \hline
     w3 = 10100100 & 1 & 1 & 0 \\
     \hline
     w4 = 1010011100 & 1 & 0 & 0 \\
     \hline
     w5 = 11110000 & 0 & 1 & 0 \\ [1ex] 
     \hline
    \end{tabular}
    \end{center}

    \item[Problem 2:] \textbf{Determining Accepted States}\\
    We have a DFA defined as follows:
    \begin{enumerate}
        \item $Q=\{q_0,q_1,q_2\}$
        \item $\Sigma=\{0,1\}$
        \item Transition function:
        \begin{align*}
        \delta(q_0,0) &= q_2, &\delta(q_0,1) &= q_0,\\
        \delta(q_2,0) &= q_2, &\delta(q_1,1) &= q_0,\\
        \delta(q_1,0) &= \delta(q_1,1) = q_1
        \end{align*}
        \item Initial state: $q_0$
        \item Accepting state: $F=\{q_1\}$
    \end{enumerate}
    For the words $w_1 = 0010$, $w_2 = 1101$, and $w_3 = 1100$, only $w_1$ and $w_2$ end in the accepted state $q_1$.
\end{description}

\subsubsection{Exploration}
I believe this material is included in the course because sets up the formal definitions of DFAs and NFAs that we can build off of later in the course. This material is quite interesting to me, because it seems you can build an automata for many things we use in our everyday life. My coffee maker has a selection of inputs, a beginning state, and a goal state, and I think it follows the definition of a DFA. Defining everyday objects as automata makes me understand coursework and comprehend the course better, as I can apply what we learn to everyday objects.

\subsubsection{Questions and Comments}
How common are objects in our lives that follow a DFA scheme? For example, an app on a phone could be seen as automata, and perhaps Instagram's "Goal State" would be keeping the app open for as long as possible.

\subsection{Week 2}
\subsubsection{Homework 2: Automata}
\begin{description}
\item[Problem 1: ] \textbf{Word Processing with DFAs}\\
Below is a table of example words and whether they are accepted by A1 and A2 as shown inside the homework.
    \begin{center}
    \begin{tabular}{||c c c||} 
     \hline
      w & A1 & A2 \\ [0.5ex] 
     \hline\hline
     w1 = AAA & 0 & 1\\
     \hline
     w2 = AAB & 1 & 0 \\
     \hline
     w3 = ABA & 0 & 0 \\
     \hline
     w4 = ABB & 0 & 0 \\
     \hline
     
     w5 = BAA & 0 & 1 \\ 
     \hline
     w6 = BAB & 0 & 0 \\
     \hline
     w7 = BBA & 0 & 0 \\
     \hline
     w8 = BBB & 0 & 0 \\ [1ex] 
     \hline
    \end{tabular}
    \end{center}

Describe the language L(A(k)) accepted by A(k) for both k=1 and k=2: \newline
A1 needs to start with an a, and then have an odd number of b's and end in a b. \newline
A2 needs to end in at least two consecutive a's.

\item[Problem 2: ] \textbf{Implementing DFA runs}\\
We begin by implementing the run method, which should begin by setting the current state to the initial state of the DFA. It then should go through all the characters in the passed in word, and update its state depending on the input and transition function. When it has finished running, it checks whether the current state is inside the list of final states and returns a boolean if it is in the list of final states. 

\begin{lstlisting}[language=Python]
  def run(self, w):
        state = self.q0
        for char in w:
            state = self.delta[(state, char)]
        if state in self.F:
            return True
        return False
\end{lstlisting}

We then move on to defining the DFAs, which can be easily defined as a set of states, a list of accepted letters, a transition function outlining what to do with each input in each state, an initial state, and finally a set of accepting states. An example is:
\begin{lstlisting}[language=Python]
  A1 = dfa.DFA(
        Q={1, 2, 3},
        Sigma={"a", "b"},
        delta={
            (1, "a"): 2,
            (1, "b"): 4,
            (2, "a"): 2,
            (2, "b"): 3,
            (3, "a"): 2,
            (3, "b"): 2,
            (4, "a"): 4,
            (4, "b"): 4,
        },
        q0=1,
        F={3},
    )
\end{lstlisting}
\item[Problem 3: ] \textbf{(Designing DFAs}\\
We implemented the DFAs into our Python program and examined what happened when we ran them with the list of all possible three letter combinations of a's and b's. All of them appeared to be correct and when we were implementing them, we saw that a lot of them shared a similar core structure and not much needed to be changed. For example, the DFA for "an odd number of a's and an odd number of b's" was identical to the DFA for "an even number of a's and an odd number of b's", except that the goal state was moved by one unit. This would apply to any even or odd counter. Another similarity was the words that contain a string of three characters, as they are identical except for the inputs.

\item[Problem 4: ] \textbf{(A new automata from an old one)}\\
To make a DFA that accepts only the words that a different DFA rejects and rejects the one that the same DFA accepts, you simply need to take the original DFA and swap the accepting states and non accepting states. For example, if A0 only accepted state 3, then A1 could accept state 1 and 2 and it would be the exact opposite. The code for this follows this premise and finds all the states that are not in the original list of accepting states.

\begin{lstlisting}[language=Python]
  def refuse(self):
        N = set()
        for state in self.Q:
            if state not in self.F:
                N.add(state)

        # Create a new DFA that uses N
        newDFA = DFA(Q=self.Q, Sigma=self.Sigma, delta=self.delta, q0=self.q0, F=N)

        return newDFA
\end{lstlisting}
\end{description}

\subsubsection{Exploration}
I think this material is included in the course because it requires the students to critically think about the formal definitions of DFAs that we can use with either pure math or graphical depictions. I think it is quite cool how DFAs can be implemented so simplistically in code, and how practical they can be. 

\subsubsection{Questions and Comments}
Is there a purpose to have the inverse of all DFAs? I understand many DFAs and their inverses have applications, but some DFAs must have an application and their inverse is completely useless

\subsection{Week 3}
\subsubsection{Homework 3: Operations On Automata}
\begin{description}
\item[Problem 1: ] \textbf{Extended Transition Functions}\\
The langauge accepted by A2 are words that are an odd number of characters in length and do not have any sequential b's. 

If we look at our function, S(state, letter), we can recursively find what the state is for any given word.
$$ S(1, abaa) = S(S(1, a), baa)$$
$$S(1, a) = 2$$
We then substitute to simplify:
$$S(2, baa) = S(S(2, b), aa)$$
$$S(2,b) = 4$$
Substitute again:
$$S(4, aa) = S(S(4, a), a)$$
$$S(4, a) = 2$$
Final State:
$$S(2, a) = 3$$

We then repeat the process for S2.
$$ S(1, abba) = S(S(1, a), bba)$$
$$S(1, a) = 2$$
We then substitute to simplify:
$$S(2, bba) = S(S(2, b), ba)$$
$$S(2,b) = 1$$
Substitute again:
$$S(1, ba) = S(S(1, b), a)$$
$$S(1, b) = 3$$
Final State:
$$S(3, a) = 3$$

\begin{figure}
    \centering
    \includegraphics[width=0.5\linewidth]{PXL_20250303_044326021.jpg}
    \caption{The $L(\mathcal A) = L(\mathcal A^{(1)}) \cap L(\mathcal A^{(1)})$}
    \label{fig:enter-label}
\end{figure}

We are sure that our new DFA, L3 is the combination of L1 and L2 because if captures every possible state of the two that is reachable. Every point in L3 is a set of the two states in L1 and L2, and it captures all possible routes that can make it to that place.

In order to change L3 such that is becomes the union of L1 and L2, you would need to make a new DFA that accepts all the words that both L1 and L2 accept. 
\end{description}

\subsection{Week 4}
\subsubsection{Homework 4: Determinization}
\begin{description}
\item[Problem 1: ] \textbf{Explain how a DFA can be seen as an NFA}\\

A DFA can be seen as an NFA because the transition function can be changed from a mapping of a state and an input to be an input and a set of states that it leads to. A DFA will have singleton states and the remaining features of the DFA stay the same in the new NFA. Because every DFA is an NFA by definition, L(A) will equal L(A') every time.

\item[Problem 2: ] \textbf{Describing NFAs. See Fig 2.}\\
\begin{figure}
    \centering
    \includegraphics[width=0.5\linewidth]{PXL_20250310_061036686.jpg}
    \caption{Homework 4, Problem 2}
    \label{fig:enter-label}
\end{figure}
\end{description}

\subsubsection{Exploration}
I like this material in the course, because I can see how an NFA would be more useful than a DFA, but harder to implement in code. I am working on the first project, and it is quite stimulating to think about the changes required to alter an NFA to become a DFA.

\subsubsection{Questions and Comments}
Are there some things that regular expressions cannot solve? It is really good at pattern matching and parsing text, but I cannot imagine that it would be good at natural language text.

\subsection{Week 5}
\subsubsection{Homework 5: ITALC Problems}
All the problems are in image form, because the proofs would take a long time to type.

\begin{figure}[h]
    \centerline{\includegraphics[width=0.5\linewidth]{ITALC 3.2.1.jpg}}
    \caption{ITALC Problem 3.2.1 Part 1}
\end{figure}

\begin{figure}
    \centerline{\includegraphics[width=0.5\linewidth]{ITALC 3.2.1.2.jpg}}
    \caption{ITALC Problem 3.2.1 Part 2}
\end{figure}

\begin{figure}
    \centerline{\includegraphics[width=0.5\linewidth]{ITALC 3.2.2.jpg}}
    \caption{ITALC Problem 3.2.2}
\end{figure}

\begin{figure}
    \centerline{\includegraphics[width=0.5\linewidth]{ITALC 4.4.1.jpg}}
    \caption{ITALC Problem 4.4.1}
\end{figure}

\begin{figure}
    \centerline{\includegraphics[width=0.5\linewidth]{ITALC 4.4.2.jpg}}
    \caption{ITALC Problem 4.4.2}
\end{figure}



\subsubsection{Exploration}
This material is quite interesting, but the proofs for RegEx are quite complicated. I am not too sure about the RegEx simplification, and I think I got some of the simplified formulas wrong. However it was very eye opening to work through these problems

\subsubsection{Questions and Comments}
Is there an easier method to prove that a DFA follows a regex structure? It takes a lot of work to probe a simple three state DFA, I cannot imagine it being feasible hand writing these formulas for more than four states. 

\subsection{Week 6 and 7}
\subsubsection{Homework 6:}
\begin{description}
\item[Exercise A:] \textbf{ }\\

1. For an input $10^n$, this Turing Machine returns $10^{n+1}$
  \begin{center}
    \begin{tabular}{||c c c c||} 
     \hline
      state & 0 & 1 & B \\ [0.5ex] 
     \hline\hline
     q0 & X & q1,1,R & X\\
     \hline
     q1 & q1,0,R & X & q2, 0, R \\
     \hline
     q2 & X & X & X \\[1ex] 
     \hline
    \end{tabular}
    \end{center}

2. For an input $10^n$, this Turing Machine returns 1
  \begin{center}
    \begin{tabular}{||c c c c||} 
     \hline
      state & 0 & 1 & B \\ [0.5ex] 
     \hline\hline
     q0 & X & q1,1,R & X\\
     \hline
     q1 & q1,B,R & X & q2, B, R \\
     \hline
     q2 & X & X & X \\[1ex] 
     \hline
    \end{tabular}
    \end{center}

3. For an input of a binary string, this returns the 1's swapped with 0's and the 0's swapped with 1's.
  \begin{center}
    \begin{tabular}{||c c c c||} 
     \hline
      state & 0 & 1 & B \\ [0.5ex] 
     \hline\hline
     q0 & q0,1,R & q1,0,R & q1,0,R\\
     \hline
     q1 & X & X & X \\[1ex] 
     \hline
    \end{tabular}
    \end{center}

\item[Exercise B: Done for Extra Credit] \textbf{ }\\
1. If we want to take input $10^n10^m$ and output $10^n10^m10^{n+m}$, we would begin by writing the entire string to the output, write an extra 1, then go back to the beginning. Then, when you read a 0, go to the end, write a 0, then go back to the beginning. When you reach a 1, go to a new 0 (using new states), and then go to the end and write a 0. Repeat this process until you hit the next 1, and then copy this pattern for the second part. You will end up with $10^n10^m10^{n+m}$.

2. In order to double any binary string, like 111000, we could have one state for each digit in the string. We could read the first digit, go to the end of the string, write the digit, go back to the first digit, then go one digit right, and repeat. This would double 111000 to 111000111000, and would double any binary string. 

\item[Exercise C: Done for Extra Credit] \textbf{ }\\
These are the transition diagrams for Exercise A of this assignment. MA1 is exercise 1, MA2 is exercise 2, and MA3 is exercise 3.
\begin{figure}[h]
    \centering
    \includegraphics[width=0.5\linewidth]{86bf363f-db99-4b51-a4ac-a5e2bc00240a~1.jpg}
    \caption{Transition Diagrams}
    \label{fig:enter-label}
\end{figure}

\end{description}

\subsubsection{Exploration}
I wonder if we can make a hardware version of a Turing machine, and could we make it more powerful than a modern processor? We may have to ditch a physical tape method, but maybe a 1s and 0s bit stream could do something similar. 

\subsubsection{Questions and Comments}
If we had a infinitely fast turing machine, what problems could we solve with it? For example, could we mine all remaining cryptocurrencies or solve the stock market? And can we create multiple constructions of turing machines that solve the same problem?

\subsubsection{Homework 7:}
\begin{description}
\item[Exercise 1:] \textbf{ }\\
Which of the following languages are decidable, recursively enumerable (r.e.), or have recursively enumerable complement (co-r.e.)?

1. $$L_1 := \{M \; | \; \text{$M$ halts on itself} \}$$
2. $$L_2 := \{(M,w) \; | \; \text{$M$ halts on the word $w$} \}$$
3. $$L_3 := \{(M,w,k) \; | \; \text{$M$ halts on $w$ in at most $k$ steps} \}$$

1. L1 is not decidable because it is essentially the diagonal langauge. This means that our language would not be able to decide in finite time whether the input is accepting or rejecting. L1 is recursively enumerable because we can take a machine that takes in Mi, and then simulates M on Mi and accepts if M accepts. If this accepts, then our machine will accept Mi. This language is not co-r.e. because the complement would include both machines that do not accept Mi and machines that do not halt on Mi, and it would be impossible for us to determine which of these machines it is. This makes it not co-r.e.

2. L2 is not decidable because there can be no Turing machine that can decide for every M and w whether M(w) halts. This was proven by Turing himself. L2 is recursively enumerable because we can simulate every step of M(w). If M(w) halts, we will see it happen, and accept, and if it doesn't halt it will run forever. L2 is not co-r.e. because if our language does not halt on the word w, we will have no way of determining if the word is in the language. This means it cannot be co-r.e.

3. L3 is decidable because there is a specific bound to check if M halted. If M has not halted within that time, we can tell that the word is not in the language. It is also recursively enumerable and decidable because it is decidable.

\end{description}

\subsubsection{Exploration}
This makes me wonder if there are a finite set of decidable and undecidable problems. I would not know how to approach this problem, but it would be interesting to see if there are infinite problems.

\subsubsection{Questions and Comments}
Is the set of decidable problems infinite? Or what about the set of undecidable problems? If they are infinite, do we have simplification methods to reduce the size of things we have to compute?
\section{Synthesis}

\section{Evidence of Participation}

\section{Conclusion}\label{conclusion}

\begin{thebibliography}{99}
\bibitem[BLA]{bla} Author, \href{https://en.wikipedia.org/wiki/LaTeX}{Title}, Publisher, Year.
\end{thebibliography}

\end{document}
